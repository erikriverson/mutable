\documentclass{article}

\usepackage{graphics}
\usepackage{lmodern}

%\VignetteIndexEntry{mutable}

\usepackage{/usr/share/R/share/texmf/Sweave}
\begin{document}


\title{Creating Markup Tables with the \texttt{mutable} Package (Version 0.0-1)}

\author{Erik Iverson}
\maketitle

\section{What is \texttt{mutable}?}

\texttt{mutable} is: 
\begin{itemize}
  \item An R package to create statistical tables ($\mu$ table)
  \item Tables are completely user-defined, and easily changed to meet the user's exact needs
  \item Tables can be exported to several different markup languages
\end{itemize}

\section{How does \texttt{mutable} differ from other table packages?}

There are already several packages on CRAN that generate
tables. Several of these even generate markup so that you can
display the resulting table in \LaTeX{} or HTML. 


\subsection{Hmisc}
Perhaps the most well-known of these functions are contained in the
\texttt{Hmisc} package, through the \texttt{summary.formula} and
\texttt{latex} functions.  I have used these functions for a long time
to generate \LaTeX{} tables for my statistical reports.  They have
served me and countless others very well, and will continue to do so.

However, I continually ran into problems when the output wasn't just
as I had hoped. There are \texttt{latex} methods for the objects
created by \texttt{summary.formula}.  The general approach is to
create an object using summary.formula, and then pass it to the latex
function. 

There are literally dozens of parameters to the latex function, many
of which control some small piece of the resulting latex output.  For
example, there is a first.hline.double parameter to control the
placment of a horizontal line, and helvetica to control the default
font used in the table. 

The main issue with this approach is that \LaTeX{} tables can be
configured to a very fine degree, and introducing a new parameter
every time you want to change a small bit of output was a painstaking
process.

Another issue related to the first was that I could not obtain an R
object with a representation of the LaTeX code that would be written
out to a file upon calling the latex function.  If this were available,
I could have finely tuned the LaTeX code 'by hand' as it were, without
introducing new parameters to the latex function. 

More substantial changes to the output required even more challenging
procedures.  When dealing with categorical variables, I find it useful
to prohibit pagebreaks in the middle of the levels of the variable.
Adding this functionality required tracking down the appropriate
places in the function to make the change, and then hoping that it
didn't break anything else.

In short, these functions take an omnibus approach to generating
markup.  If you want to alter the resulting markup code, you have to
dig into the function or introduce a new parameter.

\subsection{How does mutable fix this?}
The mutable package takes a very modular, function-based approach to
generating markup tables.  Instead of defining and setting lots of
parameters to control the final output, you pass small, user-defined
functions to generate each part of the markup. In general, the user
makes one call to mutable for each column of the resulting table.


\section{The mutable function} 

Since we will be constructing tables by making multiple calls to the
\texttt{mutable} function, it pays to describe it here.  The
\texttt{mutable} function is an S3 generic function, meaning that
multiple methods are implemented depending on the first argument.  The
two that have been implemented so far are for \texttt{formula} objects
and for \texttt{function} objects.  Let us first turn our attention to
the formula interface. 

\begin{verbatim}
mutable(formula, data, summary.function = esummary, 
        format.function = eformat, 
        latex.function = elatex, 
        html.function = ehtml, 
        colname, 
        subset = NULL, ...) 
\end{verbatim}

A summary of the arguments is given in \ref{ref:args}


\begin{table}
  \label{ref:args}
  \caption{Arguments to \texttt{mutable.formula}}
  \begin{tabular}{|p{3cm}|p{8cm}|p{3cm}|}
    \hline\\
    argument&description&default\\
    \hline\\
    formula* & an R formula, where the left-hand side is a stratification variable
    and the right-hand side is list of variables to be included as the rows in the table &
    NA\\
    &&\\
    data* & the data.frame where variables in \texttt{formula} are found & NA\\
    &&\\
    summary.function & a function to summarize each variable. Can be generic to accomodate 
    different types of data & esummary \\
    &&\\
    format.function & function that will generate plain text the object from \texttt{summary.function}
    & eformat\\
    &&\\
    latex.function & function that will generate latex markup from \texttt{summary.function} & elatex\\
    &&\\
    html.function & function that will generate HTML markup from \texttt{summary.function}& ehtml\\
    &&\\
    colname* & a character string specifying the name of the column& NA\\
    &&\\
    subset & a subset to use of \texttt{data}& NULL\\
    \hline
  \end{tabular}
\end{table}

For example, the following simple table was generated with mutable. 

\begin{Schunk}
\begin{Sinput}
> pead.bl <- data.frame(hiv = sample(c("Positive", 
+     "Negative"), 100, replace = TRUE), age = rnorm(100, 
+     c(50, 40), sd = 10), gender = sample(c("Male", 
+     "Female"), 100, replace = TRUE), diffs = rnorm(100, 
+     500, sd = 100), inc4x = sample(c("Yes", "No"), 
+     100, replace = TRUE), bmi = rnorm(100, 30, 
+     sd = 2))
> label(pead.bl$hiv) <- "HIV Status"
> label(pead.bl$age) <- "Age"
> label(pead.bl$gender) <- "Gender"
> label(pead.bl$diffs) <- "V2 GMT - V1 GMT"
> label(pead.bl$inc4x) <- "$\\frac{V1 GMT}{V2 GMT} >= 4$"
> label(pead.bl$bmi) <- "BMI"
\end{Sinput}
\end{Schunk}

We define a table by calling mutable once for each column. We combine
columns into single table object with the '+' operator. We need a
formula and a data.frame, which will proprogate from the first column
to subsequent columns if they are not specified by further columns.

Each column needs a summary.function, and functions to generate the
markup. There are defaults for these. Finally, the colname argument
must be specified.

\begin{Schunk}
\begin{Sinput}
> form <- hiv ~ age + gender + diffs + inc4x + bmi
> tab1 <- etable(form, data = pead.bl, colname = "Variable", 
+     summary.function = erownames) + etable(colname = "Combined Categories") + 
+     etable(subset = hiv == "Positive", colname = "Positive") + 
+     etable(subset = hiv == "Negative", colname = "Negative") + 
+     etable(summary.function = etest, colname = "P-value")
> latex(tab1, caption = "Baseline Table")
\end{Sinput}
\begin{table}
\begin{tabular}{lcccc}
\multicolumn{1}{c}{ Variable }&\multicolumn{1}{c}{ Combined Categories }&\multicolumn{1}{c}{ Positive }&\multicolumn{1}{c}{ Negative }&\multicolumn{1}{c}{ P-value }
\\
\hline\\ 
Age&38.5 / 44.6 / 52.2&36.3 / 44.6 / 51.1&39 / 45 / 54.9&0.20\\
Gender&&&&0.75\\
~~~ Female&45\% 45/100&48\% 22/46&43\% 23/54&\\
~~~ Male&55\% 55/100&52\% 24/46&57\% 31/54&\\
V2 GMT - V1 GMT&429 / 495.1 / 564.1&422.6 / 481 / 539.9&454.3 / 502.2 / 577.5&0.15\\
$\frac{V1 GMT}{V2 GMT} >= 4$&&&&0.53\\
~~~ No&52\% 52/100&57\% 26/46&48\% 26/54&\\
~~~ Yes&48\% 48/100&43\% 20/46&52\% 28/54&\\
BMI&28.8 / 30.2 / 31.4&28.5 / 30.5 / 31.5&28.9 / 30.1 / 31.3&0.83 \\
\end{tabular}
\caption{Baseline Table}
\end{table} \end{Schunk}

This may seem like a lot of work to generate the table, but for common
types of tables such as this, high-level wrapper functions can be 
constructed to do all the work for you.  Something else has also happened,
The tab1 object also contains an HTML version of the table.

\end{document}

